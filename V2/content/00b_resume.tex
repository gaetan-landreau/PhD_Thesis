\selectlanguage{french}
\chapter{R\'esum\'e}

\begin{mdframed}[linecolor=Prune,linewidth=1]

    \textbf{Title:} Synthèse de nouvelles vues via des considérations 3D
    
    \noindent \textbf{Mots clés:} Synthèse d'images, Reconstruction 3D, Apprentissage profond
    
    \begin{multicols}{2}
    \noindent \textbf{Résumé:} La synthèse de nouvelles vues cherche à générer des images d'une scène depuis des angles de vue non observés au préalable. Si l'avènement des techniques d'apprentissage dites \textit{profondes} a permis de réelles avancées sur ce sujet depuis 2010, le domaine conserve toujours ses anciens fondements, de la géométrie multi-vues à la reconstruction 3D. La synthèse de nouvelles vues est aujourd'hui au coeur de toutes les attentions, tant ses applications potentielles sont nombreuses, de la \ac{VR}, en passant par le rendu 3D, les jeux vidéo ou encore l'animation.

Nous abordons dans cette thèse l'une des configurations les plus contraignantes en se restreignant à n'avoir en entrée qu'une seule et unique vue.

La première partie s'intéresse à la manière dont la pose de la caméra peut être encodée et fournie comme un a priori à un \ac{NN}, via des considérations issues de la géométrie épipolaire. Cette transformation relative entre la vue d'entrée et celle à générer est souvent encodée de manière sous-optimale. Nous montrons à travers nos travaux que cette pose peut être encodée dans une image grâce aux droites épipolaires.

Nous montrons ensuite comment l'avènement récent des \ac{NeRF} a redéfini la manière d'aborder la synthèse de nouvelles vues. Ces architectures possède désormais des propriétés génératives intéressantes, qui permettent de synthétiser de nouvelles vues sans se limiter à une scène unique. Cependant, l’intégration de contraintes épipolaires dans ces réseaux reste encore peu explorée. Nous proposons donc un mécanisme d’attention simple, qui influe sur l’équation de rendu volumétrique du \ac{NeRF}.

Enfin, dans une dernière partie, nous assouplissons notre contrainte initiale pour nous rapprocher davantage d'une application industrielle. En disposant de plusieurs vues, les modèles 3D\ac{GS} permettent de reconstruire fidèlement la structure 3D d'une scène. Nous utilisons cette reconstruction 3D explicite pour synthétiser de nouvelles vues à des positions prédéterminées, éloignées des vues originellement observées lors de l'entraînement. 
\end{multicols}

\end{mdframed}
