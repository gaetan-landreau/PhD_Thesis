\selectlanguage{french}
\chapter{Extended abstract in French}
L'infographie, la reconstruction 3D et la vision par ordinateur convergent désormais lorsqu'il s'agit d'adresser le problème de la synthèse de nouvelles vues par des algorithmes d'apprentissage profond. L'idée sous-jacente est de générer de nouvelles images d'une scène à partir d'une ou plusieurs vues d'entrée. Bien que l'apprentissage profond ait considérablement fait progresser ce champ depuis 2010, ses fondements reposent encore sur des concepts bien établis depuis des décennies, notamment la géométrie multi-vues et les techniques de reconstruction tridimensionnelle. Tandis que ses applications potentielles n'ont jamais été aussi nombreuses pour l'industrie, la synthèse de nouvelles vues est désormais au centre de nombreuses recherches académiques, à la croisée de la 3D et de la génération d'images.

Dans cette thèse, nous nous intéressons à l'un des cas les plus exigeants du problème : la génération d'une nouvelle vue d'une scène statique à partir d'une unique image en entrée.

La première partie du manuscrit porte ainsi sur la manière dont la transformation de pose relative de la caméra (entre cette unique vue source et la vue cible à générer) peut être encodée et fournie à un réseau de neurones dédié à la synthèse de nouvelles vues. Nous montrons dans ce premier chapitre dans quelle mesure une telle information, souvent encodée de manière non optimale dans un vecteur latent de très basse dimension, peut être intégralement encodée dans une image, en exploitant les propriétés issues de la géométrie épipolaire. Nous présentons dans ce contexte EpipolarNVS. Cette architecture prend en entrée l'image source RGB ainsi qu'une seconde image, elle-même encodée sur trois canaux, et contenant un ensemble fini de droites épipolaires. Nous n'apportons dès lors au réseau de neurones original que le strict nécessaire de modifications pour le conditionner correctement avec notre approche, et montrons, à travers nos expérimentations, dans quelle mesure un tel encodage permet d'obtenir des vues de meilleure qualité, plus fidèles à la réalité.

Tandis que la première partie du manuscrit est intégralement consacrée à la synthèse de nouvelles vues à partir d'une unique image via des réseaux convolutifs de type \ac{CNN}, c'est vers les champs de radiance neuronaux, appelés \ac{NeRF}, que se tourne la seconde partie de ce manuscrit, toujours en ne considérant qu'une seule vue en entrée. L'idée sous-jacente à EpiNeRF, l'architecture présentée dans ce chapitre, est de chercher à capitaliser une nouvelle fois sur les propriétés issues de la géométrie épipolaire. Il est ainsi ici question d'entraîner un second \ac{NeRF} à produire des vecteurs latents alignés non pas avec l'image source mais avec l'image cible, et ce en utilisant les contraintes de la géométrie épipolaire. Par un mécanisme d'attention léger, EpiNeRF est ainsi plus efficace pour synthétiser de nouvelles vues avec des détails qui n'étaient pas nécessairement observables sur la vue source. Plusieurs expérimentations sont menées sur deux jeux de données différents pour valider cette approche.

Le dernier chapitre de ce manuscrit cherche à s'éloigner des contraintes qui étaient imposées jusqu'à présent à notre problème. L'idée de cette dernière partie est désormais de s'approcher davantage d'une application industrielle, en capitalisant sur une reconstruction 3D explicite de la scène grâce aux derniers modèles de \ac{GS}. La synthèse de nouvelles vues n'est ainsi plus obtenue à partir d'une image unique, mais à partir de plusieurs dizaines d'images d'une scène statique. Nous présentons ici un ensemble de trois améliorations, toutes ayant un rôle et un impact distinctifs sur la synthèse des vues finales. Une meilleure initialisation pour la reconstruction 3D est proposée, tout comme l'ajout d'un \ac{MLP} (qui rend dès lors l'opacité de chaque gaussienne dépendante du point de vue) pour mieux synthétiser les surfaces transparentes. Une meilleure stratégie de densification est enfin intégrée à ce système de synthèse de nouvelles vues, pour mieux restituer certains détails habituellement difficiles à conserver durant la reconstruction.


